Team formation (TF) problem in the context of social network, was first introduced by Lappas et al. in \cite{lappas2009finding}. They established that under two different communication costs, namely \textit{minimum spanning tree} (MST) and \textit{diameter} (DIA), the TF problem turns out to be NP-hard. By exploiting the connections with existing combinatorial problems such as \textit{set cover}, they developed greedy heuristics that are shown to perform well in real world datasets. In a series of follow up works \cite{anagnostopoulos2010power,kargar2011discovering,anagnostopoulos2012online,majumder2012capacitated,kargar2012efficient,kargar2013finding}, more realistic constraints and different cost functions have been studied. In \cite{kargar2011discovering}, Kargar et al proposed a version of the problem where a team of expert is selected with a \textit{leader}, and also studied it in light of two new cost functions based on sum of distances and leader distance. Later in \cite{kargar2012efficient,kargar2013finding} authors takes into account personnel costs, and formulates the bi-objective as the linear combination of communication costs and personnel costs. 

In \cite{anagnostopoulos2010power} Anagnostopoulos et al. devised an algorithm to keep the workload balanced. In \cite{anagnostopoulos2012online} they study an online version of the problem where tasks stream and once an allocation decision is made, it cannot be revoked. Majumder et al. in \cite{majumder2012capacitated}, focuses on the TF variant where workers have a maximum capacity of tasks that they can work on simultaneously. In \cite{rangapuram2013towards} authors incorporate more ``realistic'' constraints, such as team leaders, max team sizes, and geographic locality. Their formulation results in a generalized version of the ``densest subgraph problem'' with cardinality constraints, for which they then develop an approximation via a continuous relaxation. Bhowmik et al in \cite{bhowmik2014submodularity}, generalizes previous variants of TF under a single framework. They show that it's an unconstrained submodular function maximization problem, though it is non-monotone. Their generalizations include accounting for optional skills, allowing communication to go outside the team, and relaxing the constraint that the whole team be connected. More recently in \cite{farasat2016social}, authors uses TF with costs based on network structures. They rehash the Team Formation Problem from the perspective of ideas in social networks. However, none of these problems takes into account the presence of an organization. Further the tasks are deterministic. In our problem we solve the problem for an expected task requirement for an existing organization.

The wide variety in TF problem also sparked interest in benchmarking the algorithms across a unified data. In \cite{wang2015comparative} authors compare performances of different TF algorithms. 

We formulate our problem as a maximizing ratio of submodular functions. Greedy approximation that preserves $1 - \frac{1}{e}$ approximation guarantee was presented in \cite{bai2016algorithms}. 


