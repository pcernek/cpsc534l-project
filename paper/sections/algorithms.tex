We aim to solve the hiring problem with a submodular utility function and a modular cost function as such this problem becomes an instance of optimizing the ratio of monotone (sub)modular set functions. As such we base our approach in part off of \cite{bai2016algorithms}, which gives guarantees for approximation algorithms to optimize the ratio of submodular functions.
Formally, the class of problem they address has the following form:

\begin{equation}\label{prob:RS-min}
    \underset{\emptyset \subset X \subseteq V}{\min} \frac{f(X)}{g(X)}
\end{equation}

where $ f $ and $ g $ are monotone non-decreasing submodular functions.
For convenience, as in \cite{bai2016algorithms}, we use the shorthand RS (short for ratio of submodular functions) to denote any objective that has the form $ \frac{f(A)}{g(A)} $.

While \cite{bai2016algorithms} solves a minimization problem they point out that $ \min \frac{f(A)}{g(A)} = \max \frac{g(A)}{f(A)} $ and a solution for one can be used to solve the other. 

In \cite{bai2016algorithms} they additionally investigated solutions for when one or both of $f$ and $g$ are modular functions. For our purposes we use $f=\cost$ and $g=\util$, thus our problem falls into the minimization of modular $f$ and submodular $g$ 

\begin{property}
    \label{prop:mod-f-submod-g}
    Modular $ f $ and submodular $ g $
\end{property}

When $ f $ is modular and $ g $ is submodular, they show that the natural greedy algorithm (explicitly formulated as Algorithm \ref{alg:greed-ratio}) gives the following approximation:
\begin{equation}
    \label{eq:greedy-approx}
    \frac{f(\hat{X})}{g(\hat{X})} \leq \frac{e}{e - 1} \frac{f(X^*)}{g(X^*)}
\end{equation}
 
\begin{algorithm}
	\caption{\textsc{GreedRatio} for RS minimization }
	\label{alg:greed-ratio}
	\begin{algorithmic}[1]
		\renewcommand{\algorithmicrequire}{\textbf{Input:}}
		\renewcommand{\algorithmicensure}{\textbf{Output:}}
		\Require $ f $ and $ g $
		\Ensure An approximation solution $ \hat{X} $
		
		\State Initialize: $ X_0 \leftarrow \emptyset $, $ R \leftarrow V $ and $ i \leftarrow 0 $
		\While{ $ R \neq \emptyset $ }
		\State $ v \in \arg \min_{v \in R} \frac{f(v|X_i)}{g(v|X_i)} $
		\State $ X_{i+1} \leftarrow X_i \cup \left\{ v \right\} $
		\State $ R \leftarrow \left\{ v \in R | g(v|X_{i+1}) > 0 \right\} $
		\State $ i \leftarrow i + 1 $
		\EndWhile
		
		\State $ i^* \in \arg \min_i \frac{f(X_i)}{g(X_i)} $
		\State Return $ \hat{X} \leftarrow X_{i^*} $
	\end{algorithmic}
\end{algorithm}


While this GreedyRatio scheme provides an effective approximation for our variant of team formation it does not solve our optimization problem $\max \mathbb{E}[TF(G,\task)]$. This is due to that fact that the expectation of the team formation problem is no longer the ratio of submodular functions, but a non-negative linear combination of ratios of submodular functions for which there currently exists no optimization approach. 

While the problem statement is general for arbitrary hiring costs, we implement a variant dealing with unit costs. This simplifies the implementation but can be dealt with as seen in the class by running a greedy once on the ratio of the cost and the benefit and once without costs and taking the better of the two. In the case that the function being optimized is submodular this provides a better approximation guarantee. While our algorithm does not have an approximation guarantee this approach will still provide better solutions. 

This problem can be solved exactly using brute force search, however, this approach is exponential and infeasible for even modestly sized organizations and task distributions. We propose instead four approaches mixing Monte Carlo Simulation and greedy approaches: 

\begin{enumerate}
    \item Greedy + RS
    \item MC + RS
    \item Greedy + MC
    \item MC + MC
\end{enumerate}

Where first of the pair is the method used to solve the total optimization problem, and the second is the method used to solve the team formation optimization (the inner optimization).


\begin{algorithm}
	\caption{ GreedyRS Optimizaion }
	\label{alg:greed-RS}
	\begin{algorithmic}[1]
		\renewcommand{\algorithmicrequire}{\textbf{Input:}}
		\renewcommand{\algorithmicensure}{\textbf{Output:}}
		\Require Organizaiton $ H $, candidate set $ C $, budget $ B $
		\Ensure G the resulting organization after hiring
		\State $S \leftarrow \emptyset, G \leftarrow H$
		
		\While{ $C \ne \emptyset$ }
			\State $c* \leftarrow \arg \max_{c \in C} GreedyRatio(f=cost,g=util,V=G+c)$
			
			\If{ $ c_H(S)+c_H(c*) \le B $ }
				\State $S \leftarrow S + c* $
			\EndIf
			\State $C \leftarrow C - c*$
		\EndWhile
	\end{algorithmic}
\end{algorithm}

Both variants of the outer greedy could perform arbitrarily poorly due to its dependence on the curvature of the function. However, we expect in practice that this will perform well and that under certain assumptions on the functions used and the distribution of the tasks that this curvature can be bounded. Additionally this problem can be solved exactly using brute force, however, this is intractable for even modest sized problems. Instead Monte Carlo sampling is used which in the limit approaches brute force, and as such has a large dependence on the number of samples performed.  

