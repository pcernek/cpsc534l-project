We base our approach off of \cite{bai2016algorithms}, in which are given guarantees for approximation algorithms to optimize the ratio of submodular functions.
Formally, the class of problem they address has the following form:

\begin{problem}
	\label{prob:RS-min}
	\[ \underset{\emptyset \subset X \subseteq V}{\min} \frac{f(X)}{g(X)}
	\]
\end{problem}

where $ f $ and $ g $ are monotone non-decreasing submodular functions.
For convenience, as in \cite{bai2016algorithms}, we use the shorthand RS (short for ratio of submodular functions) to denote any objective that has the form $ \frac{f(A)}{g(A)} $.

\begin{property}
	\label{prop:mod-f-submod-g}
	Modular $ f $ and submodular $ g $
\end{property}

When $ f $ is modular and $ g $ is submodular, they show that the natural greedy algorithm (explicitly formulated as Algorithm \ref{alg:greed-ratio}) gives the following approximation:
\begin{equation}
	\label{eq:greedy-approx}
	\frac{f(\hat{X})}{g(\hat{X})} \leq \frac{e}{e - 1} \frac{f(X^*)}{g(X^*)}
\end{equation}
 
\begin{algorithm}
	\caption{\textsc{GreedRatio} for RS minimization }
	\label{alg:greed-ratio}
	\begin{algorithmic}[1]
		\renewcommand{\algorithmicrequire}{\textbf{Input:}}
		\renewcommand{\algorithmicensure}{\textbf{Output:}}
		\Require $ f $ and $ g $
		\Ensure An approximation solution $ \hat{X} $
		
		\State Initialize: $ X_0 \leftarrow \emptyset $, $ R \leftarrow V $ and $ i \leftarrow 0 $
		\While{ $ R \neq \emptyset $ }
		\State $ v \in \arg \min_{v \in R} \frac{f(v|X_i)}{g(v|X_i)} $
		\State $ X_{i+1} \leftarrow X_i \cup \left\{ v \right\} $
		\State $ R \leftarrow \left\{ v \in R | g(v|X_{i+1}) > 0 \right\} $
		\State $ i \leftarrow i + 1 $
		\EndWhile
		
		\State $ i^* \in \arg \min_i \frac{f(X_i)}{g(X_i)} $
		\State Return $ \hat{X} \leftarrow X_{i^*} $
	\end{algorithmic}
\end{algorithm}
 
\begin{property}
	\label{prop:submod-f-mod-g}
	Submodular $ f $ and modular $ g $
\end{property}
Conversely, for submodular $ f $ and modular $ g $, the authors leverage the relationship between RS minimization, and the minimization of the difference of submodular functions (DS). Formally, DS minimization may be described as the following optimization problem:
\begin{problem}
	\label{prob:DS-min}
	\[
	\underset{X \subseteq V}{\min} \left[ f(X) - \lambda g(X) \right]
	\]
\end{problem}
for $ \lambda \geq 0 $.
In \cite{bai2016algorithms}, it is shown that an exact algorithm for DS minimization can be used as a subroutine to solve RS minimization via a binary search scheme, as described in Algorithm \ref{alg:binary-search}.
This relationship provides the foundation for the following observation, also derived in \cite{bai2016algorithms}:
\begin{theorem}
	When $ f $ is submodular and $ g $ is modular, using an exact submodular minimization algorithm as a subroutine, Algorithm \ref{alg:binary-search} provides a $ (1 + \epsilon) $-approximation for RS minimization in $ O(\log(1/\epsilon)) $ calls to the subroutine.
\end{theorem}

\begin{algorithm}%[H]
	\caption{Binary search algorithm for RS minimization }
	\label{alg:binary-search}
	\begin{algorithmic}[1]
		\renewcommand{\algorithmicrequire}{\textbf{Input:}}
		\renewcommand{\algorithmicensure}{\textbf{Output:}}
		\Require $ f $, $ g $, $ \epsilon \in \left[0, 1\right) $, and an exact algorithm for DS minimization
		\Ensure A $ (1 + \epsilon) $-approximate solution for Problem \ref{prob:RS-min}
		
		\State Set $ \lambda_{\max} \leftarrow \frac{f(A)}{g(A)} $ for arbitrary $ A \subseteq V $, and $ \lambda_{\min} \leftarrow 0 $
		\While{ $ \lambda_{\max} > (1 + \epsilon) \lambda_{\min} $ }
		\State $ \bar{\lambda} \leftarrow \frac{\lambda_{\min} + \lambda_{\max}}{2} $
		\State $ \hat{X} \leftarrow \arg \min_{X \supset \emptyset} \left[ f(X) - \bar{\lambda} g(X) \right] $
		\If { $ \frac{f(\hat{X})}{g(\hat{X})} \geq \bar{\lambda} $ }
		\State $ \lambda_{\min} \leftarrow \bar{\lambda} $
		\Else
		\State $ \lambda_{\max} \leftarrow \bar{\lambda} $
		\EndIf
		\EndWhile
		
		\State Return $ \hat{X} \leftarrow \arg \min_X \left[ f(X) - \bar{\lambda} g(X) \right] $
	\end{algorithmic}
\end{algorithm}

Although in general, DS minimization is not tractable \cite{iyer2012algorithms}, it is pointed out in \cite{bai2016algorithms} that, under the conditions of submodular $ f $ and modular $ g $, the DS minimization problem as formulated in Problem \ref{prob:DS-min} becomes an instance of submodular minimization, and hence can be solved exactly optimally in polynomial time.

\subsection*{Implications}

Assume a modular communication cost function.
For example, the sum of the edge weights of the subgraph induced by the given team is trivially modular.

% TODO: This might be redundant with material from the Problem Statement section.
Recall that the \textsc{Top-k} problem may be formulated as follows:
\begin{problem}
	\label{prob:top-k}
	\textsc{Top-k}
	$$ \underset{X \in Candidates}{\min} \frac{cost(X)}{util(X)} $$
\end{problem}
Then, the \textsc{Top-k} problem becomes an instance of RS minimization with Property \ref{prop:mod-f-submod-g}, with $ f = cost $ and $ g = util $.
Hence, with a modular cost function, \textsc{Top-k} may be solved approximately in polynomial time by Algorithm \ref{alg:greed-ratio}, with the guarantee provided in Equation \ref{eq:greedy-approx}.

Likewise, recall that the \textsc{Bottom-k} problem may be formulated as follows:
\begin{problem}
	\label{prob:bottom-k}
	\textsc{Bottom-k}
	\[ \underset{X \in Candidates}{\min} \frac{util(X)}{cost(X)}
	\]
\end{problem}
In a similar manner, the \textsc{Bottom-k} problem becomes an instance of RS minimization with Propery \ref{prop:submod-f-mod-g}, with $ f = util $ and $ g = cost $.
It follows that a modular cost function enables \textsc{Bottom-k} to be solved approximately in polynomial time by Algorithm \ref{alg:binary-search}, with a $ (1+\epsilon) $-approximation.
