In this section, we formally define the notation and the problem we will be dealing with.
\subsection{Set-up}

\subsubsection{Person and skill}

We consider the input to our problem to include an existing organization.
We use the notation $\hiredset$ to denote the set of individuals who are already part of the organization.
Additionally, a set of candidates is denoted by $\candidates$.
Every individual person $p \in \{\hiredset \cup \candidates\}$ has a set of skills, a cost to be hired, and a cost to be included in a team (eg a node weight).
We use $\skill(p)$ to denote the skills of person $p$, $c_H(p)$ as the hiring cost, and $c_T(p)$ as the team cost.
The skills of a set of persons $\persons$ is $\skill(\persons) = \cup_{p \in \persons} \skill(p)$.

The universe of skills is denoted by the set $\skills = \{ s_1, ..., s_m \}$, where $s_i$ denotes $i$-th skill.

\subsubsection{Task Distribution}

A task $\task$ is nothing but the set of skills that are required to complete the task.
We write $\skill_i \in \task$ if $\skill_i$ is required to complete the task $\task$.

Most organizations cannot predict in advance the exact requirements of the tasks that will be requested.
We model this uncertainty using a task distribution $\tasks$. 
A task $\task \sim \tasks$ is associated with a probability.
Define $Pr[\task]$ to denote the probability with which the organization will be required to complete the task $\task$. 
%As the organization wishes to maximize the number of tasks it completes, it naturally places a higher priority on being able to complete high probability tasks.


\subsubsection{Communication Cost}
For a subset of individuals, the communication costs can be measured in different ways. 
For a set of individuals $\persons$, we denote the communication cost as $\cost(\persons)$. 
In \cite{lappas2009finding}, the diameter of the graph and minimum spanning tree (MST) are used as two different communication costs.
Many of the subsequent team formation papers study these same two measurements.
In our paper, however, we start with a simpler cost, which is just the sum of node weights. We chose the sum of node weights as our cost function as it is modular and monotone which is more amenable to analysis with existing techniques. 

\subsubsection{Utility}
Unlike communication cost, the utility of a subset of individuals depends on the task they aim to perform. 
For a task $\task$, coverage $\cov(\persons,\task)$ is the subset of skills that a team $\persons$ possess.
More formally $\cov(\persons,\task) = |\skill(\persons) \cap \task|$.
The utility of a team $ \persons $ is $\util(\persons,\task) = \frac{\cov(\persons, \task)}{|\task|}$.\\
To take into account the conflicting objectives of maximizing the utility of completing the task and minimizing the communication cost of the given team, we introduce the notion of value. We define the value of a team to be the ratio of the utility over the cost and maximize this value. Notice that maximizing the ratio of two functions is equivalent to minimizing their reciprocal. \\
We define the value of a task for a given organization (graph) $G$ as $TF(G,\task) = \max_{\persons \subseteq G}{val(\persons,\task)}$. We can now extend the notion of value to the expected value of a team over the task distribution. There we define $val(\persons,\task)$ to be $\frac{\util(\persons,\tasks)}{cost(\persons)} $ \\
Thus the expected value for the organization over the task distribution is said to be as follows.

$$\mathbb{E}[TF(G,\task)] = max_{\persons \subseteq G} \left\{ \frac{\sum_{\task \sim \tasks } Pr[\task] \times \util(\persons, \task)} {\cost(\persons)} \right\} $$ \\
Note that the numerator is a non-negative linear combination of a monotone submodular function and is also submodular.\\
Equipped with these definitions, we are now ready to formally introduce the problems that we study in this paper.

\subsection{Problems}

In this section, we present the problems we studied and analyze the connections between them. For the ease of exposition, we define the hiring problem first i.e., given a set of candidates and a hiring budget, who do we expect to be the best candidates to hire. The other problems (i.e. firing, assassination and bad hiring) are described using the same notations. We formally state each problem so that their symmetry can be clearly seen.

\subsubsection{Hiring problem}

For this problem we consider as input: an existing organization of people $\hiredset$, a candidate pool $\candidates$ from which to select new hires to the organization, and a budget $B$ that denotes the maximum amount that can be spent on hiring candidates.
We use cost and utility functions as defined above to define the problem generally as follows:
\begin{problem}
[HireMax] Given $\hiredset$, communication cost $\cost$ between every pair of individuals, a distribution of tasks $\tasks$ and a budget $B$, hire a candidate set $X$ that maximizes our organizations expected value and does not exceed our hiring budget. Thus find a $X$ such that

$$ \max \mathbb{E}[TF(G,\task)]  $$
$$ G=\hiredset  \cup X \quad c_H(X) \le B \quad X \subseteq C $$\\
Where $c_H(X) = \sum_{x \in X} c_H(x)$ is the hiring cost of the set X.\\
 
\end{problem}
    
We now show that $HireMax$ is NP-Hard.

\begin{theorem} \label{thm:HM-hardness}
HireMax is NP-Hard.
\end{theorem}

\begin{proof}
It is easy to see that to solve this problem we must solve the underlying team formation problem. Even if the underlying team formation problem is restricted to a constant cost function, this problem requires maximizing the coverage function. This is an example of submodular maximization of a monotone function under knapsack constraint, which is known to be NP-Hard.  
\end{proof}

\subsubsection{Alternative Formulations}

From the formulation of the hiring problem, three alternative problems naturally present themselves. The one which is naturally opposite to hiring is the problem of firing existing employees is presented below.  

Given an organization $\hireset$, a candidate pool $\candidates$ (often the whole organization), and a quota $Q$ we aim to fire a subset of $\candidates$ such that we free up at least $Q$ of our budget quota and we do the least damage to the organization's value. Formally find a set $X$ that, 

$$ \min \mathbb{E}[TF(G,\task)] $$
$$ G = \hiredset \setminus X \quad Q \le c_H(X) \quad  X \subseteq C $$

The third version of the problem deals with the rivalry between two organizations. An organization could be trying to assassinate some subset of the $\candidates$, from a rival organization so that the most damage is done in terms of rivals value. Again $B$ denotes the budget that can be spent on the assassination, $C$ is the candidate set that can be assassinated. Then the objective is to find a set $X$, such that

$$ \max \mathbb{E}[TF(G,\task)] $$
$$ G = \hiredset \setminus X \quad c_H(X) \le B \quad  X \subseteq C $$

Our last problem follows by the symmetry, but does not have a strong motivation, is hiring a set of candidates that adds minimum value to the organization. Here a budget quota $Q$ is given that denotes the number of bad hires that have to be made from a candidate set $C$. Thus formally the goal is to find a set $X$ such that,

$$ \min \mathbb{E}[TF(G,\task)] $$
$$ G = \hiredset \cup X \quad Q \le c_H(X) \quad  X \subseteq C $$

\subsubsection{A remark on the hardness of the alternative formulations}

Note that in all three versions we still need to solve the central team formation problem optimally. Thus the core argument of the proof of theorem \ref{thm:HM-hardness} carries forward - under constant cost we need to maximize the coverage function, which is NP-hard. Hence all these three problems are NP-hard.