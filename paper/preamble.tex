\DeclareMathOperator*{\argmin}{arg\,min}
\DeclareMathOperator*{\argmax}{arg\,max}

\newcommand{\laks}[1]{{\color{brown} [\text{Laks: } \sf #1]}}
\newcommand{\weic}[1]{{\color{blue} [\text{Wei Chen: } \sf #1]}}
\newcommand{\weil}[1]{{\color{magenta} \textsf{[Wei Lu: #1]}}}
\newcommand{\prit}[1]{{\color{red} \textsf{[Prithu Banerjee: #1]}}}
\newcommand{\red}[1]{\textcolor{black}{#1}}
\newcommand{\blue}[1]{\textcolor{black}{#1}}
\newcommand{\pink}[1]{\textcolor{black}{#1}}

\newcommand{\spara}[1]{\smallskip\noindent\textbf{#1.}}
\newcommand{\mpara}[1]{\medskip\noindent\textbf{#1.}}


\newcommand{\eat}[1]{}

% this macro is used to mark text that will only appear in the full report
\newcommand{\InFullOnly}[1]{}
 
% This is a LaTeX file
%%%%%%%%%%%%%%%%%%%%%%%%%%%%%%%%%%%%%%%%%%%%%%%%%%%%%%%%%%%%%%%%%%%%%%%%%%%%
%\makeatletter
 
%\newtheoremstyle{mythmstyle}% name of the style to be used
%  {3pt}% measure of space to leave above the theorem. E.g.: 3pt
%  {3pt}% measure of space to leave below the theorem. E.g.: 3pt
%  {}% name of font to use in the body of the theorem
%  {}% measure of space to indent
%  {}% name of head font
%  {}% punctuation between head and body
%  {}% space after theorem head; " " = normal interword space
%  {}% Manually specify head
%
%\theoremstyle{mythmstyle}
\newcommand{\nchoosek}[2]{{#1 \choose #2}}
 
\newcommand{\mybox}[1]{\vspace{5pt}\centerline{\framebox{\parbox[c]{\textwidth}{#1}}}\vspace{5pt}}
 
\newcommand{\BlackBox}{\rule{2.6mm}{2.6mm}}
\newtheorem{theorem}{Theorem}
\newtheorem{conjecture}{Conjecture}
\newtheorem{claim}{Claim}
\newtheorem{corollary}{Corollary}
\newtheorem{definition}{Definition}
\newtheorem{condition}{Condition}
\newtheorem{lemma}{Lemma}
\newtheorem{example}{Example}
\newtheorem{remark}{Remark}
\newtheorem{problem}{Problem}
\newtheorem{property}{Property}
\newtheorem{proposition}{Proposition}
\newtheorem{sub-proposition}{Sub-proposition}
\newcommand{\fillblackbox}{\hspace*{\fill}\(\BlackBox\)}
\newcommand{\fillbox}{\hspace*{\fill}\(\Box\)}
\newcommand{\fig}[1]{Figure~\ref{#1}}
\newcommand{\eqn}[1]{Equation~\ref{#1}}
\newcommand{\refsec}[1]{Section~\ref{#1}}
\newcommand{\num}[1]{(\romannumeral#1)}
% following is used to print programs within LaTeX
% \noflash{...text...} makes a box as wide as its arg, but which is whitespace
\def\noflash#1{\setbox0=\hbox{#1}\hbox to 1\wd0{\hfill}}
% \input{linespace}
 


% Macro used in the paper
%\newcommand{\shortdividerline}{\begin{center} \line(1,0){150} \end{center}}
%\newcommand{\dividerline}{\begin{center}\hrule\end{center}}
\newcommand{\non}{{\tt NULL}\xspace}
\newcommand{\U}{{\mathbb{U}}\xspace}
\newcommand{\I}{{\mathbb{I}}\xspace}
\newcommand{\seed}{{S}\xspace}
\newcommand{\D}{{D}\xspace}
\newcommand{\SW}{{SW}\xspace}
\newcommand{\adopt}{{adopt}\xspace}
\newcommand{\IN}{{inNeighbor}\xspace}
\newcommand{\util}{{util}\xspace}
\newcommand{\T}{{T}\xspace}
\newcommand{\ua}{{u}\xspace}
\newcommand{\ub}{{v}\xspace}
\newcommand{\w}{{w}\xspace}
\newcommand{\W}{{W}\xspace}
\newcommand{\price}{{price}\xspace}
\newcommand{\val}{{value}\xspace}
\newcommand{\noise}{{noise}\xspace}
\newcommand{\budget}{{B}\xspace}

%% Prithu

\newcommand{\candidate}{{c}\xspace}
\newcommand{\candidates}{{C}\xspace}
\newcommand{\hireset}{{H\xspace}}
\newcommand{\org}{{G}\xspace}
\newcommand{\nodes}{{V}}
\newcommand{\edge}{e}
\newcommand{\node}{v}
\newcommand{\edges}{E}
\newcommand{\weight}{w}
\newcommand{\hiredset}{\mathcal{H}\xspace}
\newcommand{\tasks}{\mathcal{T}}
\newcommand{\task}{T}
\newcommand{\utility}{util}
\newcommand{\gain}{\delta}




%% tight spacing in item lists
\newcommand{\squishlist}{
 \begin{list}{$\bullet$}
  {  \setlength{\itemsep}{0pt}
     \setlength{\parsep}{3pt}
     \setlength{\topsep}{3pt}
     \setlength{\partopsep}{0pt}
     \setlength{\leftmargin}{2em}
     \setlength{\labelwidth}{1.5em}
     \setlength{\labelsep}{0.5em}
} }
\newcommand{\squishlisttight}{
 \begin{list}{$\bullet$}
  { \setlength{\itemsep}{0pt}
    \setlength{\parsep}{0pt}
    \setlength{\topsep}{0pt}
    \setlength{\partopsep}{0pt}
    \setlength{\leftmargin}{2em}
    \setlength{\labelwidth}{1.5em}
    \setlength{\labelsep}{0.5em}
} }

\newcommand{\squishdesc}{
 \begin{list}{}
  {  \setlength{\itemsep}{0pt}
     \setlength{\parsep}{2pt}
     \setlength{\topsep}{2pt}
     \setlength{\partopsep}{0pt}
     \setlength{\leftmargin}{2em}
     \setlength{\labelwidth}{1.5em}
     \setlength{\labelsep}{0.5em}
    %\setlength{\itemindent}{2pt}
} }

\newcommand{\squishdesctight}{
 \begin{list}{}
  {  \setlength{\itemsep}{0pt}
     \setlength{\parsep}{0pt}
     \setlength{\topsep}{0pt}
     \setlength{\partopsep}{0pt}
     \setlength{\leftmargin}{1em}
     \setlength{\labelwidth}{1.5em}
     \setlength{\labelsep}{0.5em}
  %  \setlength{\itemindent}{2pt}
} }

\newcommand{\squishnumlist} {
\newcounter{qcounter}
\begin{list}{\arabic{qcounter}.~}{\usecounter{qcounter}} 
{  \setlength{\itemsep}{0pt}
    \setlength{\parsep}{0pt}
    \setlength{\topsep}{0pt}
    \setlength{\partopsep}{0pt}
    \setlength{\leftmargin}{2em}
    \setlength{\labelwidth}{1.5em}
    \setlength{\labelsep}{0.5em}
}}

\newcommand{\squishend}{
  \end{list}
}

% for color mark changes
\newcommand{\chgdel}[1]{\textcolor{red}{\sout{#1}}}
\newcommand{\chgins}[1]{\textcolor{blue}{#1}}

% strickout in math mode
\newcommand{\msout}[1]{\text{\sout{\ensuremath{#1}}}}
\newcommand{\chgdelm}[1]{\textcolor{red}{\msout{#1}}}
% strickout by a slash in math mode
\newcommand{\chgdels}[1]{\textcolor{red}{\cancel{#1}}}
